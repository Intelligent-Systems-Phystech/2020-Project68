\documentclass[12pt, twoside]{article}
\usepackage{jmlda}
\newcommand{\hdir}{.}

\pgfplotsset{compat=1.15}

\begin{document}

\title
    [Метаобучение тематических моделей классификации] % краткое название; не нужно, если полное название влезает в~колонтитул
    {Метаобучение тематических моделей классификации}
\author
    [А.\,С.~Ватолин] % список авторов (не более трех) для колонтитула; не нужен, если основной список влезает в колонтитул
    {А.\,С.~Ватолин, Ю.\,А.~Сердюк, К.\,В.~Воронцов} % основной список авторов, выводимый в оглавление
    [А.\,С.~Ватолин$^1$, Ю.\,А.~Сердюк$^2$, К.\,В.~Воронцов$^1$] % список авторов, выводимый в заголовок; не нужен, если он не отличается от основного
\email
    {vatolinalex@gmail.com; masyes@mail.com;  vokov@forecsys.ru}
% \thanks
    % {Работа выполнена при
    %  %частичной
	%  финансовой поддержке РФФИ, проекты \No\ \No 00-00-00000 и 00-00-00001.}
	% {Задачу поставил: Воронцов~К.\,В.
    %     Консультант: Янина~А.\,О.}
\organization
    {$^1$ Московский физико-технический институт, Москва, Россия}
\abstract
    {Одним из возможных применений вероятностной тематической модели является построение модели не только для текста, но и для 
    имеющихся методанных (модальностей). Это позволяет более точно определять темы документов, а также предсказывать 
    пропущенные методанные по имеющимся. Каждая модальность имеет свой вес, который задается вручную и отражает меру 
    влияния данной модальности на темы документов.
    В данной работе исследуются эвристики для начальной инициализации весов модальностей. Получение эвристики позволит
    полностью отказаться от перебора весов модальностей по сетке или же уменьшить количество вариантов перебора.
    Для оценки весов используется мера взаимной информации.
	
	\bigskip
	\noindent
	\textbf{Ключевые слова}: \emph {вероятностное тематическое моделирование; Мультимодальное тематическое моделирование; 
	BigARTM}
}

%данные поля заполняются редакцией журнала
% \doi{10.21469/22233792}
% \receivedRus{01.01.2017}
% \receivedEng{January 01, 2017}

\maketitle
\linenumbers

\section{Введение}
Вероятностное тематическое моделирование - способ построения модели текстовых документов, которая определяет к какой теме
относится каждый документ и какие слова образуют тему. Тематическое моделирование применяется в информационном поиске
\cite{tm_recomedation}, для классификации \cite{rubin2012statistical} и суммаризации текстов \cite{artm_summarization}, 
а также для ранжирования статей \cite{ranktopic}. Одним из продвинутых инструментов, который реализует все из перечисленных выше инструментов, является библиотека BigARTM \cite{vorontsov/artm_book}. Она обладает обширным набором параметров для настройки модели, а также различными резуляризаторами.
В данной статье внимание будет сконцентрировано на решение одной из задач, а именно - классификации.
 

Цель данной работы -- предложить эвристики для выбора оптимальной инициализации весов модальностей в тематической
модели
Тут будет введение и ссылки на статьи \cite{vorontsov/artm_book} \cite{vorontsov/blog_search}
\cite{vorontsov/transactions}  \cite{vorontsov/hierarchical}
% После аннотации, но перед первым разделом, располагается введение, включающее в себя
% описание предметной области, обоснование актуальности задачи,
% краткий обзор известных результатов.

\section{Постановка задачи}
Данный документ демонстрирует оформление статьи,
подаваемой в электронную систему подачи статей \url{http://jmlda.org/papers} для публикации в журнале <<Машинное обучение и анализ данных>>.
Более подробные инструкции по~стилевому файлу \texttt{jmlda.sty} и~использованию издательской системы \LaTeXe\
находятся в~документе \texttt{authors-guide.pdf}.
Работу над статьёй удобно начинать с~правки \TeX-файла данного документа.

Обращаем внимание, что данный документ должен быть сохранен в кодировке~\verb'UTF-8 without BOM'.
Для смены кодировки рекомендуется пользоваться текстовыми редакторами \verb'Sublime Text' или \verb'Notepad++'.

\paragraph{Название параграфа}
Разделы и~параграфы, за исключением списков литературы, нумеруются.

\section{Заключение}
Желательно, чтобы этот раздел был, причём он не~должен дословно повторять аннотацию.
Обычно здесь отмечают, каких результатов удалось добиться, какие проблемы остались открытыми.

\bibliographystyle{plain}
\bibliography{literature}

\end{document}
